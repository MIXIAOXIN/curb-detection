In this Section, we establish the preliminary notations and introduce the
mathematical models that will be used throughout the paper.

\subsection{Measurements Representation}
A nodding laser range-finder produces scan measurements $\mathbf{s}_i=[r_i,
\theta_i,\psi_i]^\text{T}$ that are transformed into their corresponding
Cartesian 3D coordinates $\mathbf{p}_i=[x_i,y_i,z_i]^\text{T}$, where $r_i$ is
a range measurement, $\theta_i$ a pitch angle, and $\psi_i$ a bearing angle. The
sensing device has an error model which is typically a function of
$\mathbf{s}_i$, i.e. $e(\mathbf{s}_i)$. From a complete laser sweep, we obtain a
point cloud representation $\mathcal{P}=\{\mathbf{p}_1,\mathbf{p}_2,\dots,
\mathbf{p}_N\}$. $\mathcal{P}$ is finally projected onto a 2D grid $\mathcal{G}=
\{\mathcal{C}_1,\mathcal{C}_2,\dots,\mathcal{C}_M\}$, with cells $\mathcal{C}_i=
\{\mathbf{c}_i,h_i,l_i,\mathcal{I}_i\}$, where $\mathbf{c}_i$ is the center of
the cell, $h_i$ its height distribution, $l_i$ its label distribution, and
$\mathcal{I}_i=\{\mathbf{p}_j\mid\mathbf{p}_j\in\mathcal{P},j=\argmin_{j'}||
\mathbf{p}_{j'}-\mathbf{c}_i||,\mathbf{p}_{j_x}<\tau_{x_{max}},\mathbf{p}_{j_x}>
\tau_{x_{min}},\mathbf{p}_{j_y}<\tau_{y_{max}},\mathbf{p}_{j_y}>\tau_{y_{min}},
\mathbf{p}_{j_z}<\tau_{z_{max}},\mathbf{p}_{j_z}>\tau_{z_{min}}\}$. The
$\tau_{[x,y,z]_{[min,max]}}$ define boundaries for the 3D points.

The posterior height distribution is a normal distribution, such that $p(h_i)=
\mathcal{N}(h_i\mid\mu_{h_i},\sigma^2_{h_i},\mathcal{I}_i)$. $\mu_{h_i}$ is the
Maximum-Likelihood Estimate (MLE) for the cell mean computed with the values
$\mathbf{p}_{j_z}$, where $\mathbf{p}_j\in\mathcal{I}_i$. $\sigma^2_{h_i}$ is
the Maximum A-Posteriori (MAP) estimate for the cell variance. The prior
distribution for $\sigma^2_{h_i}$ is an inverse gamma distribution, where we
have inserted a simplified cell error model in the hyperparameters. The label
distribution is a discrete distribution over a set of labels
$\mathcal{L}=\{1,2,\dots,M\}$.

The grid $\mathcal{G}$ will also be referred to as a Digital Elevation Map (DEM)
in the rest of the paper. The choice of a DEM representation is mainly guided by
the final outcome of the algorithm, i.e. a traversability map for the planning
process. It is also convenient for defining Regions of Interest (ROI) in
$\mathcal{P}$ and for simplifying the subsequent computations. Whenever the
number of points that fall into a cell $\mathcal{C}_i$ is below a threshold,
i.e. $|\mathcal{I}_i|<\tau_{\mathcal{I}}$, it is flagged as invalid.

\subsection{Environment Model and Inference Task}
We assume a piecewise planar environment, i.e., the observed scene is composed
of a set of plane segments. Boundaries between plane segments define local
heights discontinuities that we shall term \emph{curbs} from now on. The major
inference task therefore boils down to discovering those plane segments. To this
end, we model the environment as a \emph{mixture of linear regressions}, which
yields the following generative process for the height values:

\begin{equation}
\label{eqn:mixture}
h_i\sim p(h_i\mid\Theta)=\sum_{k=1}^K\pi_k\mathcal{N}(h_i\mid
\mathbf{w}_k^\text{T}\boldsymbol{\phi}(\mathbf{c}_i),\sigma^2_k),
\end{equation}

where $\Theta=\{\boldsymbol{\pi},\mathbf{W},\boldsymbol{\sigma}^2\}$ is the set
of adaptive parameters, $\boldsymbol{\pi}=\{\pi_k\}$ are the mixture weights,
$\mathbf{W}=\{\mathbf{w}_k\}$ the regression coefficients,
$\boldsymbol{\sigma}^2=\{\sigma^2_k\}$ the regression variances, and
$\boldsymbol{\phi}(\mathbf{c}_i)=[1,\mathbf{c}_i]^\text{T}$ is the basis
function.

Similarly to Gaussian Mixture Models (GMM), one can resort to the popular
Expectation-Maximization (EM) algorithm~\cite{dempster77maximum} for estimating
the parameter set $\Theta$ given a set of observations
$\{\{h_i,\mathbf{c}_i\}\}_{i=1}^M$. The algorithm alternates between the
computation of \emph{responsibilities} $\gamma_{ik}$ given an old estimate
$\Theta^\text{old}$, and the computation of $\Theta^\text{new}$ given
$\Theta^\text{old}$ and $\gamma_{ik}$.

In order to determine the number of planes $k$ and the initial responsibilities,
we apply the algorithm presented in Section~\ref{sec:initial}. The following
responsibilities are then evaluated with the Conditional Random Field (CRF) of
Section~\ref{sec:crf} and the parameter set $\Theta^\text{new}$ with the method
of Section~\ref{sec:plane}.
