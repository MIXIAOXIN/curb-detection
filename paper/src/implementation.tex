We have thus far delivered a formal treatment of our method to curb detection.
This section will be dedicated to algorithmic details and specific issues
arising with real-world data.

\subsection{Model Complexity and Initial Estimates}

The statistical model in \eqref{eqn:mixture} requires an estimate of the number
of mixture components $K$. This issue is related to model complexity or model
selection and remains a focus of research in itself. As our algorithm should be
be applicable to any kind of environment, a key prerequisite is to infer $K$
from the data. To this intent, we opt for an heuristic in the form of a
pre-segmentation. Specifically, we adopt the graph-based algorithm from
Felzenzswalb and Huttenlocher~\cite{felzenszwalb04efficient}. Although this
method was originally designed for image segmentation, we can adapt it for our
purpose, by treating image regions as plane segments.

The algorithm operates on the graph $\mathcal{G}=\{\mathcal{V},\mathcal{E}\}$
defined above and augment it with edge weights $w((v_i,v_j))$ proportional to
the dissimilarity between $v_i$ and $v_j$. The goal of the algorithm is to find
a partition of $\mathcal{V}$ into segments $\mathcal{S}_i$ that correspond to
the connected components of a graph $\mathcal{G}'=\{\mathcal{V},\mathcal{E}'\}$,
with $\mathcal{E}'\subseteq\mathcal{E}$. We are interested in the specific
partition such that vertices in a component have a high similarity and vertices
in different components a low similarity. Therefore, edges between vertices in
the same component should have a low weight and edges between vertices in
different components a high weight. The weight function can be defined by
the symmetric Kullback-Leibler divergence between two cells, i.e.,
$w((v_i,v_j))=D_{KL}(h_i\mid\mid h_j)+D_{KL}(h_j\mid\mid h_i)$. We thus take
into account the full height distributions, in particular the variances. For
normal distributions, the Kullback-Leibler divergence integrates analytically to

\begin{equation}
\label{eqn:kl}
D_{KL}(h_i\mid\mid h_j)=\frac{(\hat{\mu}_{h_i}-\hat{\mu}_{h_j})^2}
{2\hat{\sigma}^2_{h_j}}+\frac{1}{2}(\frac{\hat{\sigma}^2_{h_i}}
{\hat{\sigma}^2_{h_j}}-1-\ln\frac{\hat{\sigma}^2_{h_i}}{\hat{\sigma}^2_{h_j}}).
\end{equation}

The algorithm starts with all vertices belonging to a different component. It
then iterates over the set of edges ordered by increasing weights. For each
edge $(v_i,v_j)\in\mathcal{E}$ with $v_i\in\mathcal{S}_k$,
$v_j\in\mathcal{S}_l$, and $\mathcal{S}_k\neq\mathcal{S}_l$, the two components
are merged if

\begin{equation}
\label{eqn:merge}
w((v_i,v_j))\leq MInt(\mathcal{S}_k, \mathcal{S}_l),
\end{equation}

where $MInt(\mathcal{S}_k,\mathcal{S}_l)=\min(Int(\mathcal{S}_k)+
\tau(\mathcal{S}_k),Int(\mathcal{S}_l)+\tau(\mathcal{S}_l))$,
$Int(\mathcal{S})=\max_{(v_i,v_j)\in\mathcal{S}} w((v_i,v_j))$, and
$\tau(\mathcal{S})=s/|\mathcal{S}|$. Two components should be disconnected if
the difference between them is large compared to the internal difference within
at least one of the components. $s$ is a scale parameter that controls the
preference for larger components. Vertex components being segmented, an initial
estimate $\hat{\Theta}^{(0)}$ can be computed with weighted linear regression
for starting the EM algorithm.

This heuristic inevitably introduces a free parameter $s$ that controls the
model complexity. Choosing a large $s$ will potentially result in
underestimating $K$, while setting it too low in increasing the computational
complexity. In our application, we fix it at some empirical value ($s=100$) that
abates the two issues. TODO: EVALUATE THIS S!!!

\subsection{Grid Discretization and Missing Data}

The grid discretization choice is mainly influenced by the sensor
characteristics and sought accuracy. Obviously, a large cell size introduces
a proportional estimation error to the curb detection algorithm. On the other
hand, depending on the sparsity of the sensor data, a finer size increases the
number of cells with no measurement and, at the limit, results in an unconnected
graph. Indeed, whenever a cell contains no data, it is flagged as invalid and
not considered for the rest of the algorithm. Based on experimental data, we
set the cell sizes to $d_x=0.1$ [m] and $d_y=0.05$ [m], and the grid dimension
to $w=4$ [m] and $l=4$ [m].

\subsection{Algorithmic Complexity}

The algorithmic complexity depends on the aforementioned parameters ($s$, $d_x$,
$d_y$, $w$, $l$). The graph-based segmentation runs in
$O(|\mathcal{E}|\log|\mathcal{E}|)$, and the standard EM efficiently implemented
in $O(M\,K)$ with $M$ the number of valid cells. The final complexity of our
method is largely dominated by the addition of the BP pass for each E-step in
the EM. This inference method has a run-time complexity in the order of
$O(|\mathcal{E}|\,K^4)$. Nevertheless, under practical considerations and
adequate parameter setting, we can achieve close to real-time performances as
will be shown below.

\subsection{Implementation}

The algorithm has been implemented in a fully template-based C++ library that
will be available to the users at the time of publication. The belief
propagation inference engine has been borrowed to libDAI~\cite{mooij10libdai},
an open-source library for discrete approximate inference on graphical models.
