In this paper, we address the problem of curb detection for a pedestrian robot
navigating in urban environments. We devise an unsupervised method that makes
few assumptions about the environment, restricts the set of hand-tuned
parameters, and builds on sound probabilistic reasoning from the sensing device
to the outcome of the algorithm. In our approach, we construct a piecewise
planar model of the environment and determine curbs at plane segment boundaries.
Initially, we sense the environment with a nodding laser range-finder and
project the 3D measurements into an efficient Digital Elevation Map (DEM). Each
cell of the DEM maintains an error model that is propagated throughout the
entire algorithm. Plane segments are further estimated with a mixture of linear
regressions on the DEM. Here, we propose an original formulation of the standard
Expectation-Maximization (EM) algorithm for mixture models. Specifically, in the
E-step, the responsibilities are computed with a Conditional Random Field (CRF),
that smooths the assignment of DEM cells to planes of the mixture model. An
initial graph-based segmentation of the DEM provides the first responsibilities
and an estimate of the number of planes. We show promising results of the
algorithm on simulated and real-world data.
