In this paper, we address the problem of curb detection for a pedestrian robot
navigating in urban environments. We devise an unsupervised method that is
mostly view-independent, makes no assumptions about the environment, restricts
the set of hand-tuned parameters, and builds on sound probabilistic reasoning
from the input data to the outcome of the algorithm. In our approach, we
construct a piecewise planar model of the environment and determine curbs at
plane segment boundaries. Initially, we sense the environment with a nodding
laser range-finder and project the 3D measurements into an efficient Digital
Elevation Map (DEM). Each cell of the DEM maintains an error model that is
propagated throughout the entire algorithm. Plane segments are further estimated
with a mixture of linear regression model on the DEM. Here, we propose an
original formulation of the standard Expectation-Maximization (EM) algorithm for
mixture models. Specifically, in the E-step, responsibilities are computed with
a Conditional Random Field that introduces dependencies between the covariates
of the mixture model. A graph-based segmentation of the DEM provides an estimate
of the number of planes and initial parameters for the EM. We show promising
results of the algorithm on simulated and real-world data.
