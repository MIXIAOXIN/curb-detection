In this Section, we will derive the necessary steps for updating the parameter
set $\Theta$. As we shall see, this will involve the responsibilities
$\gamma_{ik}$, i.e. $p(l_i=k)$, in a weighted linear regression framework.

According to standard literature on mixture models, we may compute the weight
$\pi_k$ of the k-th plane using the sum over cell responsibilities as follows
\begin{equation}
\label{eqn:weights}
\pi_k = \frac{1}{M'}\sum_{i=1}^{M'}\gamma_{ik},
\end{equation}

where once more $M'$ corresponds to the number of valid cells in the DEM.

The MLE of the other parameters are determined by maximizing the
\emph{complete-data} log likelihood, which is denoted by
$\log p(\mathbf{h}\mid\Theta)$ and $\mathbf{h}=[h_1,h_2,\dots,h_{M'}]^\text{T}$.

In order to estimate the regression coefficients $\mathbf{w}_k$ of the k-th
plane, we furthermore introduce the diagonal matrix
$\mathbf{R}_k=\text{diag}(\gamma_{ik})$ composed of the cell responsibilities.
We may then express the regression coefficients as a function of the
responsibilities and the so-called \emph{design matrix} $\boldsymbol{\Phi}=
[\boldsymbol{\phi}(\mathbf{c}_1), \boldsymbol{\phi}
(\mathbf{c}_2),\dots,\boldsymbol{\phi}(\mathbf{c}_{M'})]^\text{T}$:

\begin{equation}
\label{eqn:coeff}
\mathbf{w}_k = (\boldsymbol{\Phi}\mathbf{R}_k\boldsymbol{\Phi})^{-1}
\boldsymbol{\Phi}^\text{T}\mathbf{R}_k\mathbf{h}.
\end{equation}

In a final step, the regression variances are modified according to

\begin{equation}
\label{eqn:coeff}
\sigma^2_k = \frac{1}{M'}\sum_{i=1}^{M'} \gamma_{ik}(h_i-\mathbf{w}_k^\text{T}
\boldsymbol{\phi}(\mathbf{c}_i))^2.
\end{equation}

The attentive reader might notice that, apart from the introduction of
$\mathbf{R}_k$, these derivations are close to the case of the single linear
regression. Here, the responsibilities act as a weight to the computations, that
is, they control the degree of influence of a particular point to the
plane regression.
