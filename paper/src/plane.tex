Given a labeling of the DEM cells belonging to the same plane segments, we show
here how we can get an estimate of the plane parameters using a Bayesian linear
regression framework.

At time step $t$ of the algorithm, we assume the following model for the plane
segment $j$

\begin{equation}
\label{eqn:regression1}
p(h_i\mid\mathbf{c}_i,\sigma^2_i,l_i,\mathbf{w}_j,\sigma^2_j)=\mathcal{N}(h_i|\mathbf{w}_j^\text{T}
\boldsymbol{\phi}(\mathbf{c}_i),\sigma^2_j).
\end{equation}

Since the $h_i$ are drawn independently, we can write the conditional likelihood

\begin{equation}
\label{eqn:regression2}
p(\mathbf{h}\mid\mathbf{C},\mathbf{w}_j,\sigma^2_j)=\prod_{i=1}^N\mathcal{N}(h_i|
\mathbf{w}_j^\text{T}\boldsymbol{\phi}(\mathbf{c}_i),\sigma^2_j),
\end{equation}

where $\mathbf{h}=[h_1,h_2,\dots,h_N]$ and $\mathbf{C}=[\mathbf{c_1},
\mathbf{c_2},\dots,\mathbf{c_N}]$.

We introduce a prior for the parameters to be estimated

\begin{equation}
\label{eqn:regression3}
p(\mathbf{w},\beta)=p(\beta)p(\mathbf{w}\mid\beta)
\end{equation}

where $p(\beta)$ is an inverse-gamma distribution and
$p(\mathbf{w}\mid\beta)$ a normal distribution.

The posterior distribution becomes

\begin{equation}
\label{eqn:regression4}
p(\mathbf{w}\mid\mathbf{h})=\mathcal{N}(\mathbf{w}\mid \mathbf{m}_N,
\mathbf{S}_N),
\end{equation}

where

\begin{eqnarray}
\label{eqn:regression5}
\mathbf{m}_N&=&\mathbf{S}_N(\mathbf{S}_0^{-1}\mathbf{m}_0+\beta
\boldsymbol{\Phi}^\text{T}\mathbf{h})\\\nonumber
\mathbf{S}_N^{-1}&=&\mathbf{S}_0^{-1}+\beta\boldsymbol{\Phi}^\text{T}\boldsymbol{\Phi}
\end{eqnarray}

and

\begin{equation}
\label{eqn:regression6}
\boldsymbol{\Phi}=
\left(
\begin{array}{cccc}
\phi_0(\mathbf{c}) & \phi_1(\mathbf{c}) & \cdots & \phi_{M-1}(\mathbf{c})\\
\phi_0(\mathbf{c}) & \phi_1(\mathbf{c}) & \cdots & \phi_{M-1}(\mathbf{c})\\
\cdots             & \cdots             & \ddots & \cdots\\
\phi_0(\mathbf{c}) & \phi_1(\mathbf{c}) & \cdots & \phi_{M-1}(\mathbf{c})\\
\end{array} \right)
\end{equation}
